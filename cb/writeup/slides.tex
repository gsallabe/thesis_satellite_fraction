\documentclass[t]{beamer}
\usetheme{default}
\usecolortheme{default}
\usepackage{macros}

\AtBeginSection[]{
\begin{frame}
    \frametitle{}
    \tableofcontents[currentsection]
\end{frame}
}


\begin{document}


\begin{frame}
    \frametitle{}
    \tableofcontents
    Christopher Bradshaw, Greg Sallaberry, Marie Wingyee Lau, Song Huang
\end{frame}


\section{Constrain the satellite fraction in HSC}
\begin{frame}
    \frametitle{Overview}
    \begin{block}{Goal}
        Constrain the high mass ($\FullMstar{} \gtrsim{} 11.5$) satellite fraction (\fsat{}) in Hyper Suprime-Cam (HSC) observations. Why HSC? Large volume -- more massive galaxies, deep images -- better \Mstar{}.
    \end{block}

    \begin{block}{How}
        \begin{itemize}
            \item In an N-body simulation, map some halo property (\eg{} \MhaloPeak{}, \vmp{}) to \Mstar{}, with some scatter.
            \item Optimize this mapping in to fit some HSC observations (\eg{} SMF, clustering).
            \item Measure \fsat{} in the best fitting mock.
        \end{itemize}
    \end{block}

    \begin{block}{See also}
        Reddick 2013 did this for SDSS
    \end{block}
\end{frame}

\begin{frame}
    \frametitle{Observations + Simulation data}

    \begin{columns}

    \column{0.5\textwidth}
    \begin{block}{Hyper Suprime Cam}
        \begin{itemize}
            \item Eventually -- 1400 deg\textsuperscript{2}, 5$\sigma$ detection to $r \approx 26$ (point sources).
            \item Data here -- 230 deg\textsuperscript{2}, $0.25 < z < 0.45$.
            \item $\sim{}$4500, 30 $\FullMstar > 11.5,\ 12$
            \item $\sim$ 95\% spec-z.
        \end{itemize}
    \end{block}

    \column{0.5\textwidth}
    \begin{block}{MDPL2}
        \begin{itemize}
            \item 1 $\Gpc{}\, \h{}$ per side
            \item Volume 40x current HSC
            \item Snapshot at $z \sim{} 0.37$
            \item $m_p = 1.5 \times 10^{9} \Msol \h$
        \end{itemize}
    \end{block}

    \end{columns}

\end{frame}

\begin{frame}
    \frametitle{Fitting choices}

    \begin{block}{Halo Parameter}
        $\Mstar{} = f({\rm halo\, property})$. We build models with \vmp{} and \MhaloPeak{}.
    \end{block}

    \begin{block}{Functional form}
        We use the 5 parameter form from Behroozi 2010 (though only 3 are needed), and a linear scatter -- $\sigma({\rm halo\, property})$.
    \end{block}

    \begin{block}{Fitting Data}
        \begin{itemize}
            \item The SMF
            \item Counts in cylinders: $\xi(r_p, r_{\pi})$ in a single $r_p < 1\,\Mpc$ and $r_{\pi} < 10\,\Mpc$ bin. HSC doesn't have enough data for a more detailed measurement of clustering. This is a cross correlation between galaxies $\FullMstar > M_{cut}$ and $M_{cut} - 0.1 < \FullMstar < M_{cut}$.
        \end{itemize}
    \end{block}

\end{frame}

\begin{frame}
    \frametitle{Fits with \Mhalo{}}

    \begin{columns}
    \column{0.5\textwidth}

    \begin{block}{SMF}
        \begin{figure}
        \includegraphics[width=\textwidth]{images/fit_smf_mhalo.png}
        \end{figure}
    \end{block}


    \column{0.5\textwidth}
    \begin{block}{Clustering}
        \begin{figure}
        \includegraphics[width=\textwidth]{images/fit_clust_mhalo.png}
        \end{figure}
    \end{block}
    \end{columns}
\end{frame}

\begin{frame}
    \frametitle{Bestfit Models 1: SMF}

    \begin{columns}
    \column{0.5\textwidth}

    \begin{block}{\vmp{}}
        \begin{figure}
        \includegraphics[width=\textwidth]{images/fit_smf.png}
        \end{figure}
    \end{block}


    \column{0.5\textwidth}
    \begin{block}{\MhaloPeak{}}
        \begin{figure}
        \includegraphics[width=\textwidth]{images/fit_smf_mpeak.png}
        \end{figure}
    \end{block}
    \end{columns}

\end{frame}

\begin{frame}
    \frametitle{Bestfit Models 2: Clustering}

    \begin{columns}
    \column{0.5\textwidth}

    \begin{block}{\vmp{}}
        \begin{figure}
        \includegraphics[width=\textwidth]{images/fit_clust.png}
        \end{figure}
    \end{block}


    \column{0.5\textwidth}
    \begin{block}{\MhaloPeak{}}
        \begin{figure}
        \includegraphics[width=\textwidth]{images/fit_clust_mpeak.png}
        \end{figure}
    \end{block}
    \end{columns}
\end{frame}

\begin{frame}
    \frametitle{Results 1: \fsat{}}

    \begin{figure}
    \includegraphics[width=0.9\textwidth]{images/sat_frac.png}
    \caption{Errors on the UM are statistical from the bestfit. Errors on our mocks include the uncertainty of the parameters.}
    \end{figure}
\end{frame}

\begin{frame}
    \frametitle{Questions}
    \begin{itemize}
        \item{Can we use X-ray observations to measure \fsat{} in observations?}
        \item{Are there other ways in which centrals can be distinguished from satellites? Shape, orientation, something else?}
    \end{itemize}
\end{frame}


\section{Modelling RSD in DESI}
\begin{frame}
    \frametitle{Modelling RSD in DESI with the Universe Machine?}

    \begin{block}{The Universe Machine (UM)}
        \begin{itemize}
            \item{Semi-empirical model that computes $SFR(\vmax{}, z, \dot{V}_{\rm max})$}
            \item{Fits many observations: SMF, clustering, CSFR, etc}
        \end{itemize}
    \end{block}

    \begin{figure}
    \includegraphics[width=0.9\textwidth]{other_images/um_params.png}
    \end{figure}
\end{frame}

\begin{frame}
    \frametitle{Parameter Reduction?}

    Currently there is huge amount of flexibility in the UM model that we might not need, particularly in SFR at high $z$ and in quenching physics.

    Peter estimates that $10-15$ params could be removed in those areas.
\end{frame}

\begin{frame}
    \frametitle{Resolution Requirements}

    We need to resolve the satellites that build up the large centrals.

    Rodriguez-Gomez estimates that the bulk of $\FullMstar > 11.4$ galaxies comes from $\FullMstar > 10.4$ (that mostly grow from in-situ star formation).

    90\% of these galaxies are in halos with $\FullMhalo > 11.5$ at $z = 0$. However these need to be tracked at higher redshifts. Peter estimates that the UM needs to resolve $\FullMhalo > 10.5$.

    What $m_p$ can we use?

\end{frame}


\begin{frame}
    \frametitle{Questions}
    \begin{itemize}
        \item{N-body resolution requirements for the UM to do RSD?}
        \item{How many UM params can be cut?}
        \item{Baryonic effects?}
    \end{itemize}
\end{frame}


\end{document}

% \item I don't use any covariances in my best fit. This is certainly wrong though unclear how important? Not important in SMF.
%       Should use auto rather than cross correlation to minimize in clustering.

Better label/legends in fsat plot
understand the redick model

New way sof modelling RSD with DESI. after best fit - RSD with desi. Can we use UM to model RSD.

UM params (big list).
Which ones can we remove. (see google doc)
* Resolution requirements
* Number of params, what can be cut
